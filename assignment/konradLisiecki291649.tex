

%----------------------------------------------------------------------------------------
%	PACKAGES AND OTHER DOCUMENT CONFIGURATIONS
%----------------------------------------------------------------------------------------

\documentclass[oneside]{book} 
\usepackage{graphicx}
\usepackage{tabularx}
\usepackage{amsmath,amssymb} 
\usepackage{fancyhdr}
\usepackage{listings}  
\usepackage{hyperref}
\usepackage[margin=1in]{geometry}
\usepackage[svgnames, table]{xcolor} % Required to specify font color

\usepackage{fancyhdr}
\usepackage[utf8]{inputenc}

% do tabel
\usepackage[T1]{fontenc}
\usepackage[utf8]{inputenc}
\usepackage{tabularx,ragged2e,booktabs,caption}


\usepackage{color}

\definecolor{mygreen}{rgb}{0,0.6,0}
\definecolor{mygray}{rgb}{0.5,0.5,0.5}
\definecolor{mymauve}{rgb}{0.58,0,0.82}

\lstset{ %
  backgroundcolor=\color{white},   % choose the background color; you must add \usepackage{color} or \usepackage{xcolor}
  basicstyle=\footnotesize,        % the size of the fonts that are used for the code
  breakatwhitespace=false,         % sets if automatic breaks should only happen at whitespace
  breaklines=true,                 % sets automatic line breaking
  captionpos=b,                    % sets the caption-position to bottom
  commentstyle=\color{mygreen},    % comment style
  deletekeywords={...},            % if you want to delete keywords from the given language
  escapeinside={\%*}{*)},          % if you want to add LaTeX within your code
  extendedchars=true,              % lets you use non-ASCII characters; for 8-bits encodings only, does not work with UTF-8
  frame=single,                    % adds a frame around the code
  keepspaces=true,                 % keeps spaces in text, useful for keeping indentation of code (possibly needs columns=flexible)
  keywordstyle=\color{blue},       % keyword style
  language=Octave,                 % the language of the code
  morekeywords={*,...},            % if you want to add more keywords to the set
  numbers=left,                    % where to put the line-numbers; possible values are (none, left, right)
  numbersep=5pt,                   % how far the line-numbers are from the code
  numberstyle=\tiny\color{mygray}, % the style that is used for the line-numbers
  rulecolor=\color{black},         % if not set, the frame-color may be changed on line-breaks within not-black text (e.g. comments (green here))
  showspaces=false,                % show spaces everywhere adding particular underscores; it overrides 'showstringspaces'
  showstringspaces=false,          % underline spaces within strings only
  showtabs=false,                  % show tabs within strings adding particular underscores
  stepnumber=2,                    % the step between two line-numbers. If it's 1, each line will be numbered
  stringstyle=\color{mymauve},     % string literal style
  tabsize=2,                       % sets default tabsize to 2 spaces
  title=\lstname                   % show the filename of files included with \lstinputlisting; also try caption instead of title
}

\newcolumntype{C}[1]{>{\Centering}m{#1}}
\renewcommand\tabularxcolumn[1]{C{#1}} 



\makeatletter 
\makeatother

\newcommand*{\plogo}{\fbox{$\mathcal{PL}$}} % Generic publisher logo

%----------------------------------------------------------------------------------------
%	TITLE PAGE
%----------------------------------------------------------------------------------------

\newcommand*{\titleTH}{\begingroup % Create the command for including the title page in the document
\raggedleft % Right-align all text
\vspace*{\baselineskip} % Whitespace at the top of the page

{\Large Konrad Lisiecki 291649}\\[0.167\textheight] % Author name 

{\LARGE\bfseries Zadanie Zaliczeniowe}\\[\baselineskip] % First part of the title, if it is unimportant consider making the font size smaller to accentuate the main title

{\textcolor{Red}{\Huge Obliczeniowa Teoria Wyboru Społecznego}}\\[\baselineskip] % Main title which draws the focus of the reader

{\Large \textit{Semestr zimowy  2014/15  }}\par % Tagline or further description

\vfill % Whitespace between the title block and the publisher

{\large  Warszawa, 2015-02-12}\par % Publisher and logo
\renewcommand{\chaptername}{Podpunkt}

\vspace*{3\baselineskip} % Whitespace at the bottom of the page
\endgroup}


\newcolumntype{R}{>{\raggedleft\arraybackslash}X}
%----------------------------------------------------------------------------------------
%	BLANK DOCUMENT
%----------------------------------------------------------------------------------------

\begin{document} 

\pagestyle{empty} % Removes page numbers
	

\titleTH % This command includes the title page

 
%\listoffigures
%\listoftables

%\chapter*{Wprowadzenie}
%\addcontentsline{toc}{chapter}{Wprowadzenie}

\chapter*{Rozwiązanie}
 
Celem zadania jest znalezienie algorytmu na znalezienia wartości 	:
\begin{equation}
contribution(\alpha_i, VBS(A))
\end{equation}
Zgodnie ze wzorem jest rowne:
\begin{equation}
contribution(\alpha_i, VBS(A)) =  \phi_i((A,v))
\end{equation}
Przypomnijmy, że zgodnie
 z definicja z zadania
 wartość shapleya 
 dla gracza $\alpha_i \in A$ wynosi 
 \begin{equation}
 \phi_i((A,v)) = \frac{1}{n!}\sum_{} \Delta_{\pi}^G (\alpha_i)
 \end{equation} 
 czyli gdy podstawimy do wzour zamiast $v$ $performance$ otrzymamy:
  \begin{equation}
 \phi_i((A,v)) = \frac{1}{n!}\sum_{} \Delta_{\pi}^G (\alpha_i)
 \end{equation} 
Co można rozpisać również w następujący sposób :
  \begin{equation}
 \phi_i((A,v)) = \frac{1}{n!}\sum_{\pi \in \Pi^A} ( performance(VBS(C_i^{\pi} \cup \{a_i\})) -  performance(VBS((C_i^{\pi})) )
 \end{equation}  
   \begin{equation}
 \phi_i((A,v)) =\frac{1}{n!} \frac{1}{ |X|}    \sum _{x \in X} \sum_{\pi \in \Pi^A} (   min_{a_i \in C_i^{\pi}  \cup \{a_i\}} time(a_i, x)  - min_{a_i \in C_i^{\pi}  } time(a_i, x) )
 \end{equation} 
 Co ten wzór oznacza? Oznacza on, ze aby otrzymać wartość Shapleya dla algorytmu $a_i$ możemy rozpatrywać każde zadanie z osobna.
 Czyli teraz głównym problemem jest to jak efektywnie obliczyć przyrost efektywności (zbioru algorytmu) z przejściem algorytmu $a_i$ dla każdej permutacji dla danego zadania, czyli jak efektywnie obliczyć ten człon:
  \begin{equation}
 \sum_{\pi \in \Pi^A} (   min_{a_i \in C_i^{\pi}  \cup \{a_i\}} time(a_i, x)  - min_{a_i \in C_i^{\pi}  } time(a_i, x) )
 \end{equation} 
 
 
 W tym celu (dla danego) zadania uszeregujmy algorytmy według ich wydajności (szybkości dojścia do rozwiązania) np: dla zadania $x_3$ mamy następującą kolejność:
  $$  x_3:  a_1, a_2, a_3, a_4, a_5, a_6, a_7 $$
  
 Zauważmy, ze jeżeli algorytm $a_i$ jest najwolniejszy ze wszystkich to nie wniesie żadnego przyrostu efektywności do jakiegokolwiek podzbioru algorytmu.
 
 Zauważmy, tez ze jeżeli algorytm $a_i$ dochodzi do zbioru w którym już jest szybszy od niego algorytmy to tez nie wniesie żadnej poprawy efektywności.
 
 Jak zatem będziemy obliczać wartość Shapley dla algorytmu $a_i$ dla konkretnego zadania. Prześledźmy to na krok po kroku. Załóżmy, że $a_i$ to $a_4$ z przykładu wyżej.
 

Algorytm będzie wnosił przyrost efektywności tylko, gdy koalicja do której dochodzi nie zawiera szybszego algorytmu, a więc składnik wartości Shapleya będzie niezerowy dla:
\begin{enumerate}
\item wszystkich permutacji w których $a_5$ jest najszybszy, a tych jest (i to mnożymy razy przyrost efektywności pomiędzy $a_4$, a $a_5$):
\item[•] $(3-1)! {2 \choose 2} = 2$ dla zbiorów 3-elementowych 
\item[•] $(2-1)! {2 \choose 1} = 2$ dla zbiorów 2-elementowych 
\item[•] $(1$ dla zbiorów 1-elementowych   

\item wszystkich permutacji w których $a_6$ jest najszybszy, a tych jest (i to mnożymy razy przyrost efektywności pomiędzy $a_4$, a $a_6$):
\item[•] $(1-1)!  {1 \choose 1}= 1$ dla zbiorów 2-elementowych 
\item[•] $(1$ dla zbiorów 1-elementowych   
\item wszystkich permutacji w których $a_7$ jest najszybszy, a tych jest(i to mnożymy razy przyrost efektywności pomiędzy $a_4$, a $a_7$):
\item[•] $1$ dla zbiorów 1-elementowych   
\end{enumerate}


Gdy zsumujemy te wszystkie iloczyny, wystarczy to podzielić przez $n!$ oraz $|X|$, aby otrzymać wartość Shapleya dla danego algorytmu.

Rozwiązanie to działa w czasie wielomianowym.
 

\end{document}






 